%%%%%%%%%%%%%%%%%
% This is an sample CV template created using altacv.cls
% (v1.1.5, 1 December 2018) written by LianTze Lim (liantze@gmail.com). Now compiles with pdfLaTeX, XeLaTeX and LuaLaTeX.
%
%% It may be distributed and/or modified under the
%% conditions of the LaTeX Project Public License, either version 1.3
%% of this license or (at your option) any later version.
%% The latest version of this license is in
%%    http://www.latex-project.org/lppl.txt
%% and version 1.3 or later is part of all distributions of LaTeX
%% version 2003/12/01 or later.
%%%%%%%%%%%%%%%%

%% If you need to pass whatever options to xcolor
\PassOptionsToPackage{dvipsnames}{xcolor}

%% If you are using \orcid or academicons
%% icons, make sure you have the academicons
%% option here, and compile with XeLaTeX
%% or LuaLaTeX.
% \documentclass[10pt,a4paper,academicons]{altacv}

%% Use the "normalphoto" option if you want a normal photo instead of cropped to a circle
% \documentclass[10pt,a4paper,normalphoto]{altacv}

\documentclass[10pt,a4paper,ragged2e]{altacv}

%% AltaCV uses the fontawesome and academicon fonts
%% and packages.
%% See texdoc.net/pkg/fontawecome and http://texdoc.net/pkg/academicons for full list of symbols. You MUST compile with XeLaTeX or LuaLaTeX if you want to use academicons.

% Change the page layout if you need to
\geometry{left=1cm,right=9cm,marginparwidth=6.8cm,marginparsep=1.2cm,top=1.25cm,bottom=1.25cm}

% Change the font if you want to, depending on whether
% you're using pdflatex or xelatex/lualatex
\ifxetexorluatex
  % If using xelatex or lualatex:
  \setmainfont{Carlito}
\else
  % If using pdflatex:
  \usepackage[utf8]{inputenc}
  \usepackage[T1]{fontenc}
  \usepackage[default]{lato}
  \usepackage[hidelinks]{hyperref}
\fi

\usepackage{hyperref}

% Change the colours if you want to
\definecolor{GoodColor}{HTML}{571946}
\definecolor{SlateGrey}{HTML}{2E2E2E}
\definecolor{LightGrey}{HTML}{666666}
\definecolor{Good2}{HTML}{7442c8}
\colorlet{heading}{GoodColor}
\colorlet{accent}{Good2}
\colorlet{emphasis}{SlateGrey}
\colorlet{body}{LightGrey}

% Change the bullets for itemize and rating marker
% for \cvskill if you want to
\renewcommand{\itemmarker}{{\small\textbullet}}
\renewcommand{\ratingmarker}{\faCircle}

%% sample.bib contains your publications
\addbibresource{sample.bib}

\begin{document}
\name{Alexey Ischenko}
\tagline{Software Engineer}
\personalinfo{%
  % Not all of these are required!
  % You can add your own with \printinfo{symbol}{detail}
  \email{\href{mailto:ishhenko.as@phystech.edu}{ishhenko.as@phystech.edu}}
  \phone{+79774691438}
  \github{\href{https://github.com/ale110i}{ale110i}}
  \location{Moscow, Russia}
  %% You MUST add the academicons option to \documentclass, then compile with LuaLaTeX or XeLaTeX, if you want to use \orcid or other academicons commands.
  % \orcid{orcid.org/0000-0000-0000-0000}
}

%% Make the header extend all the way to the right, if you want.
\begin{fullwidth}
\makecvheader
\end{fullwidth}

%% Depending on your tastes, you may want to make fonts of itemize environments slightly smaller
% \AtBeginEnvironment{itemize}{\small}

%% Provide the file name containing the sidebar contents as an optional parameter to \cvsection.
%% You can always just use \marginpar{...} if you do
%% not need to align the top of the contents to any
%% \cvsection title in the "main" bar.
\cvsection[page1sidebar]{Experience}

\cvevent{Middle Software Developer}{\href{https://yandex.com/company}{Yandex LLC}}{11.2021 -- ...}{Moscow, Russia}
Yandex Monitoring distributed systems
\begin{itemize}
    \item DevOps
    \item System Administrator
    \item C++, Java
\end{itemize}
\bigskip

\cvevent{Intern Software Developer}{\href{https://yandex.com/company}{Yandex LLC}}{08.2021 -- 10.2021}{Moscow, Russia}
Yandex Database distributed systems
\begin{itemize}
    \item C++
\end{itemize}
\bigskip

\cvevent{Teaching Assistant in Discrete Mathematics}{\href{https://www.hse.ru/en/info/}{Higher School of Economics}}{2020 -- ...}{Moscow, Russia}
\begin{itemize}
    \item Help students master the course
    \item Grade students’ homework assignments
\end{itemize}

% \cvevent{\href{https://github.com/ale110i/GraphRealization}{Unified Simple Graph Interface (C++)}}{\href{https://mipt.ru/english/about/about-mipt/}{Moscow Institute of Physics and Technology}}{2018 -- 2019}{Dolgoprudny, Russia}
% Common interface for matrix, list, set and arc graphs, also realised:
% \smallskip
% \begin{itemize}
%     \item \href{https://github.com/ale110i/PrimsAlgorithm}{Prim's algorithm}
%     % \item \href{https://en.wikipedia.org/wiki/Kruskal\%27s_algorithm}{Kruskal's algorithm}
%     % \item \href{https://en.wikipedia.org/wiki/Bellman–Ford_algorithm}{Bellman-Ford algorithm}
%     % \item \href{https://en.wikipedia.org/wiki/Kosaraju\%27s_algorithm}{Kosaraju's algorithm}
%     \item \href{https://github.com/ale110i/HashTableOptimum}{Template Class HashTable}
%     \item \href{https://github.com/ale110i/MinimalGraphCycle}{Minimal Cycle Detector}
% \end{itemize}

% \bigskip

% \cvevent{\href{https://github.com/ale110i/KnuthMorrisPratt}{String-Searching Module (C++)}}{\href{https://www.hse.ru/en/info/}{Higher School of Economics}}{2019 -- 2020}{Moscow, Russia}
% \begin{itemize}
%     % \item \href{https://en.wikipedia.org/wiki/Aho–Corasick_algorithm}{Aho-Corasick algorithm}
%     \item \href{https://en.wikipedia.org/wiki/Knuth–Morris–Pratt_algorithm}{Knuth-Morris-Pratt algorithm}
% \end{itemize}

% \bigskip

% \cvevent{National \& Phystech Olympiad Organizer}{\href{https://mipt.ru/english/about/about-mipt/}{Moscow Institute of Physics and Technology}}{2017 -- 2019}{Moscow, Russia}
% Organized events for high school students:
% \smallskip
% \begin{itemize}
%     \item To motivate them to apply to universities
%     \item To provide an ability to enhance their portfolio
% \end{itemize}

\cvsection{Education}

\cvevent{\textbf{\href{https://www.hse.ru/en/ba/se/about/}{BS in Software Engineering}}}{\href{https://www.hse.ru/en/info/}{Higher School of Economics}}{2020 -- June 2024}{Moscow, Russia}
Department of Computer Science\\
\smallskip
Speciality: Software Engineering\\
\smallskip
Main courses:
\smallskip
\begin{itemize}
    \item Algorithms \& Data Structures
    \item Machine Learning \& Neural Networks
    \item Cross-Platform Applications
    \item C/C++, C\#, Java, Python, Assembler
    \item Calculus, Linear Algebra, Discrete Math
\end{itemize}

\bigskip

\cvevent{\textbf{\href{https://mipt.ru/english/edu/departments/diht}{BS in Computer Science} (Not finished)}}{\href{https://mipt.ru/english/about/about-mipt/}{Moscow Institute of Physics and Technology}}{2017 -- 2019}{Dolgoprudny, Russia}
Department of Innovation and High Technology\\
\smallskip
Speciality: Applied Mathematics and Information Technology\\
\smallskip
Main courses:
\smallskip
\begin{itemize}
    \item Algorithms \& Data Structures
    \item Machine Learning \& Neural Networks
    \item C/C++, SQL, Java, Python
    \item Git, Bash, Linux
    \item Calculus, Linear Algebra, Discrete Math, Graph Theory
\end{itemize}

\end{document}
